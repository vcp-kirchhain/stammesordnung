\documentclass[a4paper]{article}

\usepackage[left=1.5cm, right=1.5cm, top=1.5cm, bottom=1.5cm]{geometry} 
\usepackage[utf8]{inputenc} 
\usepackage{german} 
\usepackage{amsmath} 
\usepackage{amssymb} 
\usepackage{listings} 
\usepackage{hyperref} 
\usepackage{graphicx} 
\usepackage{eurosym} 
\usepackage{ulem} \DeclareUnicodeCharacter{20AC}{\euro}

% \usepackage{helvet}
\renewcommand{\familydefault}{\sfdefault}

\begin{document}

\begin{center}
    \begin{minipage}{.2\textwidth}
        \flushleft
        \includegraphics[width=1\linewidth]{Lutherrose.pdf}
    \end{minipage}%
    \begin{minipage}{.6\textwidth}
        \begin{center}
            \footnotesize Verband Christlicher Pfadfinderinnen und Pfadfinder\\
            \large Stamm Martin Luther Lumdatal\\
            ~\\
            \Large \textbf{Änderungsantrag zur Stammesordnung}\\
            \normalsize Anpassung der Ordnung an die gelebte Praxis\\
            ~\\
        \end{center}
    \end{minipage}%
    \begin{minipage}{.2\textwidth}
        \flushright
        \includegraphics[width=.85\linewidth]{Zeichen.pdf}
    \end{minipage}%
\end{center}
~\\[0.5cm] 

\emph{Die Mitgliederversammlung möge beschließen die Stammesordnung wie folgt zu ändern:}

\section{Anpassung der Ordnung in den folgenden Punkten:} % (fold)
\label{sec:anpassung_der_ordnung_auf_die_neue_beauftragung}

\begin{table}[h]
\def\arraystretch{2}
\center
\begin{tabular}{ l|p{.45\textwidth}|p{.45\textwidth}}
\# & \textbf{Stammesordnung 2014}                                                                         & \textbf{geänderte Version}                                                                                                                                                                                                                 \\ \hline
1  & ... Stammeswappen ist nach dem Namensgeber des Stammes die Lutherrose.                               & ... Stammeswappen ist nach dem Namensgeber des Stammes die Lutherrose. Sie ist auf der Titelseite dieses Dokumentes abgebildet.                                                                                                               \\ \hline
2  & Jede Sippe besitzt einen Sippenführer, der für die Planung und Durchführung ... zuständig ist.       & Jede Sippe besitzt mindestens einen Sippenführer, der oder die für die Planung und Durchführung ... zuständig sind.                                                                                                                        \\ \hline
3  & (nicht vorhanden)                                                                                    & 4.1.1 Aufgaben der Führungsrunde... Betrachtung der langfristigen Planung für den Stamm                                                                                                                                                    \\ \hline
4  & (nicht vorhanden)                                                                                    & 4.2.1 Stammesführer... Es ist möglich zwei Personen als Stammesführer einzusetzen, die beiden Stammesführer treten dann als Team gegenüber der MV auf. Die Stammesführer koordinieren sich selbst und teilen die Aufgaben unter sich auf. \\ \hline
5  & (nicht vorhanden)                                                                                    & 3.3 Ranger/Rover Stufe... Die Aufnahme in die Ranger/Rover Stufe erfolgt mit Erneuerung des Pfadfinderversprechens vor dem Stamm. Die Führungsrunde kann sich ein individuelles Ritual zur Aufnahme überlegen.                                                                                                        \\ \hline
6  & Die Führungsrunde wird vom Stammesführer einberufen und sollte mindestens alle 2 Monate stattfinden. & Die Führungsrunde wird vom Stammesführer einberufen und sollte mindestens einmal im Monat stattfinden.                                                                                                                                    
\end{tabular}
\end{table}
% section anpassung_der_ordnung_auf_die_neue_beauftragung (end)
~\\[0.5cm]
{\emph{Begründung zum Antrag: }
Seit der Beschluss dieser Ordnung 2011 haben sich Kleinigkeiten in der Praxis der Stammesarbeit geändert. Dieser Antrag dient dazu die Ordnung auf Realität im Stamm abzugleichen. 
\paragraph{3} Die Langfristige und Nachhaltige Planung der Stammesarbeit ist sehr wichtig und sollte deshalb gesondert erwähnt werden.
\paragraph{5} Der Änderungsantrag zur Anpassung an die Stufenkonzeption hat das Aufnahmeritual der Stufen mitgebracht, weswegen auch hier das Ritual erwähnt werden sollte.  


\end{document}






