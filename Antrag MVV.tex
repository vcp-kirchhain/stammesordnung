\documentclass[a4paper]{article}

\usepackage[left=1.5cm, right=1.5cm, top=1.5cm, bottom=1.5cm]{geometry} 
\usepackage[utf8]{inputenc} 
\usepackage{german} 
\usepackage{amsmath} 
\usepackage{amssymb} 
\usepackage{listings} 
\usepackage{hyperref} 
\usepackage{graphicx} 
\usepackage{eurosym} 
\usepackage{ulem} \DeclareUnicodeCharacter{20AC}{\euro}

% \usepackage{helvet}
\renewcommand{\familydefault}{\sfdefault}

\begin{document}

\begin{center}
    \begin{minipage}{.2\textwidth}
        \flushleft
        \includegraphics[width=1\linewidth]{Lutherrose.pdf}
    \end{minipage}%
    \begin{minipage}{.6\textwidth}
        \begin{center}
            \footnotesize Verband Christlicher Pfadfinderinnen und Pfadfinder\\
            \large Stamm Martin Luther Lumdatal\\
            ~\\
            \Large \textbf{Änderungsantrag zur Stammesordnung}\\
            \normalsize Einführung eines Mitgliederversammlungsvorstandes (MVV)\\
        \end{center}
    \end{minipage}%
    \begin{minipage}{.2\textwidth}
        \flushright
        \includegraphics[width=.85\linewidth]{Zeichen.pdf}
    \end{minipage}%
\end{center}
~\\[1cm] 

\emph{Die Mitgliederversammlung möge beschließen die Stammesordnung wie folgt zu ändern:}

\section{Hinzufügen von Absatz 4.4: Mitgliederversammlungsvorstand (MVV)} % (fold)
\label{sec:hinzufugen_von_absatz_4_4_mitgliederversammlungsvorstand}
	\begin{itemize}
		\item Der MVV besteht aus zwei Personen, die volljährig, Mitglied im VCP und für keine andere Aufgabe durch die MV beauftragt sind. 
		\item Die Amtszeit beträgt 2 Jahre, nach Möglichkeit sollte jedes Jahr eine Person neu gewählt werden. Wiederwahl ist möglich. 
		\item Der Mitgliederversammlungsvorstand lädt die Mitgliederversammlung ein und stellt die Tagesordnung zusammen mit dem Stammesführer auf. 
		\item Der MVV holt die Berichte der verschiedenen Beauftragten ein und stellt sie der Mitgliederversammlung zur Verfügung. 
		\item Der Vorstand moderiert die Mitgliederversammlung, führt Protokoll und stellt dieses dem Stamm und der Öffentlichkeit innerhalb von vier Wochen zur Verfügung. 
		\item Der MVV wacht über die Beschlüsse der MV und deren Einhaltung. 
	\end{itemize}

\section{Anpassung der Ordnung auf die neue Beauftragung} % (fold)
\label{sec:anpassung_der_ordnung_auf_die_neue_beauftragung}

\begin{table}[h]
\def\arraystretch{1.3}
\center
\begin{tabular}{ p{.45\textwidth}|p{.45\textwidth}}
\textbf{Stammesordnung 2014}                                                                                                                                              & \textbf{geänderte Version}                                                                                                                             \\ \hline
Die Mitgliederversammlung tritt regulär einmal im Jahr zusammen und wird durch den Stammesführer einberufen und moderiert.                                       & Die Mitgliederversammlung tritt regulär einmal im Jahr zusammen.                                                                              \\\hline
(nicht vorhanden)                                                                                                                                                & Die Mitgliederversammlung ist das höchste beschlussfassende Organ des Stammes und entscheidet damit in letzter Instanz innerhalb des Stammes. \\\hline
Die Mitgliederversammlung wählt einen Stammesführer, einen Kassenwart, einen Materialwart, mindestens einen Kassenprüfer und weitere Stammesbeauftragte.         & Die Mitgliederversammlung wählt die in dieser Ordnung beschriebenen Beauftragungen.                                                           \\\hline
Die Mitgliederversammlung nimmt den Bericht des Stammesführers, der Führungsrunde, des Kassenwartes, der Kassenprüfer und weiteren Stammesbeauftragten entgegen. & Die Mitgliederversammlung nimmt die Berichte der Beauftragen entgegen, der MVV kümmert sich um die Einholung derer.                          
\end{tabular}
\end{table}
% section anpassung_der_ordnung_auf_die_neue_beauftragung (end)
~\\[0.5cm]
{\emph{Begründung zum Antrag: }
Demokratie beruht auf dem Prinzip der Gewaltenteilung. Im Moment ist der Stammesführer dafür Verantwortlich, die Mitgliederversammlung einzuberufen und zu leiten, des weiteren Leitet er die Führungsrunde und ist in der Stammesleitung. Um dem Stammesführer nicht alle Macht im Stamm zu übergeben und den Stammesführer zu entlasten, soll der Mitgliederversammlungsvorstand (MVV) eingesetzt werden. 

Die Personen aus dem MVV dürfen kein anderes Amt im Stamm bekleiden, damit die Trennung der Verantwortlichkeiten bestehen bleibt. Damit die MV die wesentlichen Punkte der Stammesarbeit abdeckt stellt der MVV die Tagesordnung zusammen mit dem Stammesführer auf.

Die Ordnung eines Vereins ist für die schlechten Zeiten geschrieben und als Absicherung für die Mitglieder gedacht. Sie soll nicht dazu führen in die aktive Arbeit zu erschweren.



\end{document}






